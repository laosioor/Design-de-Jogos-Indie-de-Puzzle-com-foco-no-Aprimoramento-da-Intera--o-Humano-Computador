\setlength{\absparsep}{18pt} % ajusta o espaçamento dos parágrafos do resumo
\begin{resumo}
	
	A indústria de videogames consolidou-se como uma das mídias mais rentáveis e complexas da atualidade, exigindo interfaces que garantam uma boa experiência ao usuário. Este trabalho situa-se na interseção entre a Interação Humano-Computador (IHC) e o \textit{Design} de Jogos, com foco específico nos jogos independentes de gênero \textit{puzzle}. O problema central abordado é o desafio de ensinar regras e mecânicas ao jogador sem subestimar sua inteligência ou quebrar a imersão com tutoriais excessivamente explícitos. A hipótese levanta que a construção de uma experiência eficaz depende da integração harmoniosa dos elementos de Game \textit{Design} — estética, mecânica, história e tecnologia — para guiar o jogador de forma intuitiva. O objetivo geral é analisar como implementações de \textit{design} em jogos de \textit{puzzle} influenciam, positiva ou negativamente, a percepção e interação dos jogadores. A metodologia adotada é de caráter qualitativo, consistindo em uma revisão bibliográfica das principais teorias da área, seguida pela coleta de dados através da experimentação e análise crítica de estudos de caso de jogos selecionados. Como resultado, espera-se desenvolver diretrizes práticas e recomendações para \textit{designers}, demonstrando que o favorecimento da descoberta orgânica das regras gera maior imersão e uma percepção mais positiva da interface do que métodos instrucionais tradicionais.
	
	\vspace{\onelineskip}
	
	% todas em letras minúsculas, separadas por ponto e vírgula (;)
	\textbf{Palavras-chave}: Design de Jogos; Interação Humano-Computador; Jogos de Puzzle; Jogos Indie; Experiência de Usuário.
	
\end{resumo}