\chapter{Considerações Gerais}
\label{cap:02}

\section{Interação Humano-Computador}

\subsection{Definição}

A definição da Interação Humano-Comptuador surge a partir da idéia de que: ''[\dots] the user and the computer engage in a communicative dialogue whose purpose is the accomplishment of some task'' \citeauthorandyear{Card1983}.

Os autores apontam que o processo de se projetar uma interação é feita de forma emergente, isto é, ocorre a partir do processo da criação, usabilidade e testes de uma ferramenta.

Décadas antes de sua definição, a atividade de se imputar e receber informações de um sistema ocorria com a necessidade de três pessoas, como os autores apontam:
\begin{itemize}
	\item 1º: o usuário, a pessoa que queria realizar alguma tarefa utilizando um sistema;
	\item 2º: o perfurador de cartões, a pessoa que iria utilizar cartões perfurados para traduzir o problema do usuário para conseguir ser lido por meio desses cartões no sistema;
	\item 3º: por fim, um operador de computadores inseria-lhe o(s) cartão(ões) perfurado(s) para o leitor de cartões do computador.
\end{itemize}
Após todo esse longo processo, o computador então, leria o cartão, e caso estivesse tudo certo, imprimiria mensagens que seriam entregues ao usuário para que este pudesse descobrir se sua tarefa foi realizada ou não com sucesso.


% vou escrever mais, obviamente, apenas anotei coisas enquanto estava lendo...

\subsection{Exemplos}
Texto da subseção de exemplos de IHC

\subsection{Cenário Atual}
Texto da subseção do Cenário Atual de IHC

\section{Jogos}

\subsection{Definição}
Definir um jogo é uma tarefa complicada e subjetiva, desta forma ao olharmos para trabalhos literários que tentam este feito, chegamos a \citeauthorandyear{Crawford1984} que define jogos de uma forma abrangente, exemplificando desde jogos de cartas, como também jogos de tabuleiro e atléticos. Além disso, o autor diz que: ``a game is a closed, formal system that subjectively represents a subset of reality.''\footnote{``Um jogo é um sistema formal e fechado que representa subjetivamente um subconjunto da realidade'' - Tradução nossa.}
De forma detalhada, o ator aponta que:
\begin{itemize}
\item Um jogo ser fechado significa que este é um produto completo e independente em sua estrutura;
\item Sua formalidade é representada nas suas regras explicitas. Para que algo aconteça tal coisa precisa ser feita previamente, por exemplo;
\item Um jogo é também um sistema, visto que é uma aplicação computacional, isto é, um conjunto de códigos que ao serem executados são transformados em uma experiência interativa.
\end{itemize}

% vou escrever mais, obviamente, apenas anotei coisas enquanto estava lendo...

\subsection{Gêneros}
Texto da subseção de Gêneros de Jogos

\section{Design de Jogos}
\citeauthorandyear{Schell2008} acrescenta em sua definição de videogames, como tendo quatro elementos principais: Estética, Mecânicas, História e Tecnologia. Parte do papel do \textit{Game Designer} durante o planejamento de um jogo é definir como todos esses elementos são interligados enquanto o jogamos:

\begin{itemize}
\item A estética molda o sentimento passado pelo jogo. Caso seja uma experiência focada no terror, a estética deve evocar uma sensação de tensão e medo. Por outro lado, um jogo familiar deve trazer o sentimento de conforto enquanto joga, por exemplo;
\item A mecânica são todos os sistemas e regras contidos dentro do jogo, como \textit{puzzles}\footnote{\textit{Puzzle} é o termo em inglês para um quebra-cabeça, ou seja, um problema a ser resolvido com o uso de lógica de acordo com as regras situadas em seu contexto, podendo também ser usado para designar um gênero de jogos focados na resolução de problemas de lógica.}, combates, movimentação, exploração etc. São definidores do que jogadores podem ou não fazer, e o que ocorre caso eles tentem;
\item A história é o esqueleto narrativo do jogo, pode ser linear e sequencial como é geralmente em livros e filmes, mas também pode ser aberto a escolhas tendo múltiplos caminhos;
\item A tecnologia é tudo aquilo usado durante a parte de desenvolvimento de um jogo, inclusive papel e caneta como denota \citeauthorandyear{Schell2008}. De forma geral, são os computadores, consoles, programas e tudo que leva a ideia até sua conclusão.
\end{itemize}

\section{Exemplos do \LaTeX}

Texto da revisão da literatura, dividido em seções e subseções.

Este é um exemplo de como usar figuras. Referência cruzada: Figura~\ref{fig:exemplo}

\FloatBarrier
\begin{figure}[!htbp]
	\centering
	\caption{Exemplo de figura}
	%scale redimensiona a figura.
	%1.5 = 150% do tamanho original
	%1 = 100% do tamanho original
	%0.20 = 20% do tamanho original
	\includegraphics[scale=0.4]{imagens/exemploFigura}
	\\\textbf{Fonte:} Elaborada pelo autor
	\label{fig:exemplo}
\end{figure}
\FloatBarrier


Este é um exemplo de como usar tabelas. Referência cruzada: Tabela~\ref{tab:exemplo}

\FloatBarrier
\begin{table}[!htbp]
\centering
\caption{Exemplo de tabela de 2 colunas}
	\begin{tabular}{ c | c }
		\hline
		\textbf{Coluna 1} & \textbf{Coluna 2} \\ \hline
		Dado 1a           & Dado 2a           \\ \hline
		Dado 1b           & Dado 2b           \\ \hline
		Dado 1c           & Dado 2c           \\ \hline
		Dado 1d           & Dado 2d           \\ \hline
	\end{tabular}
	\\ \vspace{0.2cm}
	\textbf{Fonte:} Elaborada pelo autor
	\label{tab:exemplo}
\end{table}
\FloatBarrier


Este é um exemplo de como usar quadros. Referência cruzada: Quadro~\ref{qua:exemplo}

\FloatBarrier
\begin{quadro}[!htbp]
	\centering
	\caption{Exemplo de quadro}
	\includegraphics[scale=.7]{imagens/exemploQuadro}
	\\\textbf{Fonte:} Elaborada pelo autor
	\label{qua:exemplo}
\end{quadro}
\FloatBarrier


Este é um exemplo de como usar equações. Referência cruzada: Equação~\ref{eq:exemplo}

\begin{equation}
\sum_{i=1}^{n} i = \frac{n(n+1)}{2}
\label{eq:exemplo}
\end{equation}


Exemplo de inserção de lista de código fonte:

\lstinputlisting[language=Java]{fontes/ClasseExemplo.java} 



Este é um exemplo de como inserir texto sem formatação (ambiente verbatim):

\begin{verbatim}
	Texto sem formatação, como espaçamento igual.
\end{verbatim}


Exemplo de lista de itens:

\begin{itemize}
	\item \textbf{Item 1:} texto...;
	\item \textbf{Item 2:} texto...;
    \begin{itemize}
            \item \textbf{Subitem:} texto...;
            \item \textbf{Subitem:} texto...;
            \item \textbf{Subitem:} texto...;
        \end{itemize}
	\item \textbf{Item 3:} texto...;
	\item \textbf{Item n:} texto....
\end{itemize}


Exemplo de lista numerada:

\begin{enumerate}
	\item \textbf{Item:} texto...;
	\item \textbf{Item:} texto...;
    \begin{enumerate}
        \item \textbf{Subitem:} texto...;
        \item \textbf{Subitem:} texto...;
        \item \textbf{Subitem:} texto...;
    \end{enumerate}
	\item \textbf{Item:} texto...;
	\item \textbf{Item:} texto....
\end{enumerate}


Exemplos de comandos para texto e referências:

\begin{itemize}
	\item Para iniciar um novo parágrafo, basta deixar uma linha em branco no código fonte;
	\item Não force o compilador a pular mais de uma linha, pois terá influência negativa na composição do documento;
	\item Sempre deixe o \LaTeX\ realizar a formatação de parágrafos e posicionamento de elementos;
	\item Utilização de aspas simples (abertura \verb|`|, fechamento \verb|'|): `Texto entre aspas simples';
	\item Utilização de aspas duplas (abertura \verb|``|, fechamento \verb|''|): ``Texto entre aspas duplas'';
	\item Negrito (comando \verb|\textbf|): \textbf{texto em negrito};
	\item Itálico (comando \verb|\textit|): \textit{texto em itálico};
	\item Sublinhado (comando \verb|\underline|): \underline{texto sublinhado};
	\item Negrito e itálico (usar comandos juntos): \textbf{\textit{texto em negrito e itálico}};
	\item Alterar cor do texto (comando \verb|\textcolor{cor}{texto}|):
	\begin{itemize}
		\item Exemplo \verb|\textcolor{red}{texto}|: \textcolor{red}{texto vermelho};
		\item Exemplo \verb|\textcolor[RGB]{255, 102, 0}|: \textcolor[RGB]{255, 102, 0}{texto laranja};
		\item Exemplo \verb|\textcolor[HTML]{006AD7}|: \textcolor[HTML]{006AD7}{texto azul};
	\end{itemize}
	\item Ambiente matemático inline (comando \verb|$ expressão $|): $s = x^2-2x +1$;
	\item Referência normal (comando \verb|\cite|):
	\begin{itemize}
		\item \cite{Agaisse1995};
		\item \cite{Abedi2014};
		\item \cite{BtNomenclature2016};
	\end{itemize}
	\item Referência normal com mais de uma obra (comando \verb|\cite|):
	\begin{itemize}
		\item \cite{Abedi2014, Agaisse1995};
		\item \cite{AgapitoTenfen2014, BtNomenclature2016, Nelson2014};
	\end{itemize}
	\item Referência nome e ano (comando \verb|\citeauthorandyear|):
	\begin{itemize}
		\item \citeauthorandyear{Agaisse1995};
		\item \citeauthorandyear{Abedi2014};
		\item \citeauthorandyear{BtNomenclature2016};
	\end{itemize}
\end{itemize}


Exemplo 1 de citação direta:

\begin{citacao}
	Os 20 aminoácidos usualmente encontrados como resíduos em proteínas contém um grupo $\alpha$-carboxil, um grupo $\alpha$-amino e um grupo R distinto substituído no átomo de carbono $\alpha$. O átomo de carbono $\alpha$ de todos os aminoácidos, com exceção da glicina, é assimétrico e, portanto, os aminoácidos podem existir em pelo menos duas formas estereoisoméricas. Somente os estereoisômeros L, com uma configuração relacionada à configuração absoluta da molécula de referência L-gliceraldeído, são encontrados em proteínas \cite[p. 81]{Nelson2014}.
\end{citacao}

Exemplo 2 de citação direta:

\begin{citacao}
	\textit{These various insecticidal proteins are synthesized during the stationary phase and accumulate in the mother cell as a crystal inclusion which can account for up to 25\% of the dry weight of the sporulated cells. The amount of crystal protein produced by a B. thuringiensis culture in laboratory conditions (about 0.5 mg of protein per ml) and the size of the crystals (24) indicate that each cell has to synthesize $10^6$ to $2 \times 10^6$ $\delta$-endotoxin molecules during the stationary phase to form a crystal} \cite[p. 1]{Agaisse1995}.
\end{citacao}

Exemplo de nota de rodapé\footnote{Essa é uma nota de rodapé!}.


\section{Trabalhos Correlatos}

Pesquise e descreva no mínimo três trabalhos correlatos ao seu.

\subsection{Trabalho 1}

Texto...

\subsection{Trabalho 2}

Texto...

\subsection{Trabalho 3}

Texto...
