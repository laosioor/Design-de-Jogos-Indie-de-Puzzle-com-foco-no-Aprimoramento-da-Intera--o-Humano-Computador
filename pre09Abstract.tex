\setlength{\absparsep}{18pt} % ajusta o espaçamento dos parágrafos do resumo
\begin{resumo}[Abstract]
	
	\begin{otherlanguage*}{english}
		
		The video game industry has established itself as one of the most profitable and complex media forms today, requiring interfaces that ensure a good user experience. This work is situated at the intersection of Human-Computer Interaction (HCI) and Game Design, specifically focusing on the indie puzzle genre. The central problem addressed is the challenge of teaching rules and mechanics to the player without underestimating their intelligence or breaking immersion with overly explicit tutorials. The hypothesis posits that constructing an effective experience depends on the harmonious integration of Game Design elements—aesthetics, mechanics, story, and technology—to guide the player intuitively. The general objective is to analyze how design implementations in puzzle games influence, positively or negatively, player perception and interaction. The methodology adopted is qualitative, consisting of a bibliographic review of the main theories in the field, followed by data collection through experimentation and critical analysis of case studies of selected games. As a result, it is expected to develop practical guidelines and recommendations for designers, demonstrating that favoring the organic discovery of rules generates greater immersion and a more positive perception of the interface compared to traditional instructional methods.
		
		\vspace{\onelineskip}
		 
		\textbf{Keywords}: Game Design; Human-Computer Interaction; Puzzle Games; Indie Games; User Experience.
		
	\end{otherlanguage*}

\end{resumo} 