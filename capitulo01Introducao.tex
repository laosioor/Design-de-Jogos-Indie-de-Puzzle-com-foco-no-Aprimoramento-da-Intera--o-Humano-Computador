\chapter{Introdução}
\label{cap:01}

Desde a década de 1980, quando computadores passaram a ter interfaces gráficas, cientistas da computação observaram a necessidade intrínseca de desenvolver interfaces competentes para auxiliar no entendimento de seus sistemas, gerando uma boa experiência para aqueles que os utilizam, sendo chamados de usuários. A partir disso ocorre o início de estudos a respeito da Interação Humano-Computador (IHC), uma área multidisciplinar da computação que preza principalmente pelo estudo de sistemas e a forma como eles funcionam, além de como os usuários interagem, reagem e se comunicam com tais interfaces.

A computação por meio de experimentação, afere-se com as teorias dentro da psicologia partindo de um pressuposto inicial de que programas desenvolvidos com um propósito específico capturam dados relativos à sua execução. Como \citeauthorandyear{Oliveira2015} destacam, estes dados se tornam posteriormente uma ferramenta para confirmar ou invalidar teorias. Essas teorias, por sua vez, orientam analistas de sistemas no desenvolvimento de interfaces.

Graças aos aparatos dessas evoluções tecnológicas, uma grande nova arte foi criada: os videogames, no início eram usados apenas como uma forma de entretenimento e lazer, porém com o tempo foram se engrandecendo e se tornando cada vez mais complexos, de forma que hoje em dia contam com histórias, aventuras e ideias grandiosas, sendo considerada a mídia mais rentável atualmente, conforme \citeauthorandyear{Pavlik2008} comenta ao discutir sobre o crescimento deste mercado.

A partir de 2008, artigos já contavam sobre o enorme sucesso dos jogos eletrônicos no mercado, difundindo sobre como eles estavam se tornando mais rentáveis do que outras mídias como filmes e músicas. Como \citeauthorandyear{Pavlik2008} divulgou: A indústria de videogames cresceu exponencialmente atingindo o marco de 30 bilhões de dólares de vendas anuais em todo o mundo, gerando mais receitas anualmente do que Hollywood.

Dentro da produção de um jogo, existe uma etapa essencial chamada de \textit{Game Design}, na qual \citeauthorandyear{Schell2008} destrincha sobre o assunto e suas peculiaridades em seu livro, que consiste no processo de formulação de toda parte criativa, técnica e de experiência final do usuário. Tal etapa é designada geralmente a um indivíduo, chamado de \textit{Game Designer}, que tem em seu foco cuidar da visão geral do projeto do início ao fim para que nada fique destoante de sua concepção original.

Videogames são uma arte interativa na qual o jogador passa por uma experiência, projetada seguindo alguma intenção por trás, para no fim gerar determinadas emoções e sentimentos pensados anteriormente pelo \textit{Game Designer}. Como afirmado por \citeauthorandyear{Salen2003}, muitos passos são envolvidos durante esse processo no desenvolvimento de um jogo. A usabilidade e acessibilidade são aspectos essenciais durante o planejamento desta etapa, garantindo que jogadores possam interagir de forma eficaz com sua interface. É importante estudar o \textit{feedback} dos usuários e suceder por meio de testes de usabilidade. \citeauthorandyear{Schell2008} enfatiza que essa etapa é realizada quando os desenvolvedores reúnem pessoas para avaliar sessões de seu jogo e analisar como cada participante interage com seus aspectos, investigando se alcançam os resultados esperados durante a concepção das ideias e aperfeiçoando-os.

\citeauthorandyear{Barbosa2010} apontam a importância no conhecimento de abordagens, métodos e técnicas de IHC devido a influência que um bom \textit{design} impacta na qualidade do produto final. Além disso há mérito em analisar casos de sucesso e de insucesso de interfaces com o usuário, procurando de alguma forma comprovar os motivos que levaram a tal resultado. Nisso torna-se evidente a semelhança de ambos processos de coleta e análise de dados de usuários, tanto na criação de jogos, quanto na criação de interfaces de sistemas.

\section{Justificativa}

Dado tal contexto, ao se analisar como a IHC e o \textit{design} de jogos se comunicam de forma natural, e como estudo dos mesmos são imprescindíveis para a criação de bons sistemas, torna-se evidente a importância de se compreender as nuances das implementações de \textit{Game Design}.	

Assim sendo, é levantada a questão: como as implementações de \textit{Game Design} em diferentes jogos pode influenciar de forma positiva ou negativa a experiência de seus jogadores? Além disso, questiona-se quais são as boas práticas e conceitos utilizados por \textit{Designers} de Jogos?

Dentro do gênero de \textit{puzzle}, em particular, esses questionamentos se tornam ainda mais críticos. O desafio central para o \textit{Designer} do Jogo é ensinar as regras e possibilidades de um sistema ao jogador sem subestimar sua inteligência ou quebrar a imersão. Tutoriais excessivamente explícitos podem transformar a descoberta, que é o cerne da experiência de um quebra-cabeça, em uma simples execução de tarefas. Em contrapartida, uma abordagem de \textit{design} que se apoia na intuição e permite que o jogador aprenda organicamente através da própria interação com o jogo é frequentemente associada a uma experiência mais gratificante e imersiva.

Cria-se então a hipótese de que, a construção de uma experiência de usuário eficaz em jogos de \textit{puzzle} independentes não depende da complexidade das mecânicas, mas sim da forma como os elementos de \textit{Game Design} (estética, mecânica, história e tecnologia)  são integrados para guiar o jogador intuitivamente. Portanto, presume-se que jogos com \textit{design} que favorecem a descoberta orgânica das regras, em detrimento de tutoriais explícitos, geram maior imersão e uma percepção mais positiva da interface.


\section{Objetivos}

\subsection{Objetivo Geral}

Desta forma, o objetivo geral deste trabalho se encontra na intersecção entre os estudos sobre a Interação Humano-Computador e o \textit{Design} de Jogos, buscando analisar implementações de \textit{Game Design} em jogos \textit{indie}\footnote{\textit{Indie} vem de independente, sendo denominado por um produto ou estilo cultural criado e distribuído sem financiamento de grandes empresas.} do gênero \textit{puzzle}, para, a partir disso, desenvolver um protótipo de jogo de \textit{puzzle} que aplique e valide os conceitos estudados. Para alcançar tal finalidade encontra-se os seguintes objetivos específicos:

\subsection{Objetivos Específicos}
\begin{itemize}
	\item Realizar uma revisão da literatura sobre as principais teorias e práticas de \textit{Game Design} com foco em jogos de \textit{puzzle}, construindo assim uma base teórica sobre estratégias de \textit{design} que têm demonstrado eficácia;
	\item Definir requisitos e diretrizes de \textit{design} para construção de um jogo de \textit{puzzle} com base nos resultados da pesquisa;
	\item Implementar um protótipo jogável que incorpore as diretrizes elaboradas, focando na aplicação prática dos conceitos de \textit{Design} de Jogos e Interação Humano-Computador.
\end{itemize}