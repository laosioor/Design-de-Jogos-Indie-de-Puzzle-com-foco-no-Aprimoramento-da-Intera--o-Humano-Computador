\chapter{Metodologia}
\label{cap:03}

Este projeto utilizará uma gama de atividades para garantir sua conclusão. As etapas necessárias para o desenvolvimento dessas atividades serão descritas na ordem em que ocorrerão no diagrama apresentado na Figura~\ref{fig:metodologia}:

\begin{figure}[!htbp]
    \centering
    \caption{Diagrama da Metodologia do Projeto}
    \includegraphics[scale=0.4]{imagens/metodologia_projeto.png}
    \\\textbf{Fonte:} Elaborada pelo autor
    \label{fig:metodologia}
\end{figure}

Esse diagrama tem um sentido de leitura ocidental, ou seja, deve ser lido da esquerda para a direita, de cima para baixo. Sua primeira atividade está descrita como “Revisão Bibliográfica” e a última como “Resultados e Disseminação”. Além do fluxo principal decorrendo cada uma das futuras atividades, também pode se notar alguns círculos pintados de azul, ligados por linhas tracejadas às atividades principais. Estes círculos identificam tópicos e sub-atividades que serão explorados dentro de cada uma das atividades às quais estão conectados. Nas próximas Seções, o objetivo de cada uma das atividades será detalhado. 

\section{Revisão Bibliográfica}

A metodologia terá início com uma revisão bibliográfica abarcando livros e artigos que se relacionam com o tema do projeto, visando adquirir conhecimento abrangente de forma prévia sobre \textit{Design} de Jogos, do gênero \textit{puzzle}, e identificar como ideias surgem para a criação de elementos e mecânicas, observando também as possíveis dificuldades durante o desenvolvimento de um jogo quando tratando de engajar uma boa experiência final para o usuário. Assim, formulando possíveis critérios de quais são os pontos definidores de boas decisões dentro do planejamento de um jogo de \textit{puzzle}. 

A pesquisa será feita usufruindo de plataformas que integram artigos científicos, como a plataforma de periódicos da CAPES, a biblioteca digital da Association of Computing Machinery (ACM) e da Institute of Electrical and Electronics Engineers (IEEE); o Google Acadêmico; e a plataforma Lattes, além de bibliotecas físicas e virtuais, como a biblioteca Pearson, concedidas pelo próprio IFSP. 

\section{Coleta de Dados}
Selecionando os jogos a serem analisados, a coleta de dados terá início, utilizando métodos descritivos e interpretativos que definem a pesquisa como sendo de caráter qualitativo. Nesta etapa visa-se fundamentar o estudo de caso, sendo realizado a partir de experimentações e comprovações das teorias estudadas na fase de revisão bibliográfica. Para isso, serão selecionados jogos nos quais o pesquisador avaliará e compreenderá os feitos com base em seus triunfos, mais tarde durante a análise de dados, conforme os critérios estabelecidos durante a pesquisa. 

Durante a coleta de dados, as ferramentas que se farão indispensáveis serão um computador e um console, ambos sendo de posse do pesquisador, e os jogos necessários serão adquiridos por meio do próprio pesquisador, contemplando lojas virtuais como a Steam ou a Epic Games, sem a necessidade de financiamento externo. 

\section{Análise de Dados}

A análise de dados irá mensurar todo o conhecimento e experiências que serão adquiridas tanto na revisão da literatura, como também na coleta de dados pelos métodos já descritos. Cada jogo será avaliado de forma sucinta utilizando das métricas que serão estudadas e definidas, durante o levantamento literário, compreendendo e avaliando os feitos com base em seus triunfos e falhas e buscando entender o motivo por trás de suas criações, além dos resultados que eram esperados, durante o seu desenvolvimento.